\chapter{Conclusions}
\label{chap:conclusions}
This thesis presented Biohadoop, a new framework to build bio-inspired optimization techniques executable on Apache Hadoop. Two different GAs were implemented on top of Biohadoop. Their performance was evaluated on a Hadoop cluster with 6 dual-core computers that were connected to a \unit[1]{Gb} Ethernet network.

The results show that algorithms implemented with Biohadoop have the potential to scale efficiently if the parallel part of the problem dominates the whole execution time. An example of such a problem is the tiled matrix multiplication (TMM). It provided speedups of up to 10 when compared to the execution with less resources (one worker). ZDT-3 as the other benchmark exposed a behavior that lead to poor speedups of about 2. The problem with this benchmark was that the parallel running parts were very short in terms of computational time. The sequential time and communication overhead dominated the whole runtime.

The execution time comparison with standalone, sequential implementations demonstrated further advantages of Biohadoop. TMM provided in this case speedups of about 8. In contrast, the performance of ZDT-3 was worse to the point that the parallel execution took longer than the sequential version.

In conclusion, Biohadoop demonstrated to be a useful framework for the implementation of bio-inspired optimization techniques on Hadoop. It provides a powerful non-blocking API that simplifies the design of distributed applications. Hadoop as solid and well tested basis provides the necessary cluster management and reliable application execution. Further, the evaluation results show that algorithms suited to parallelization and implemented with Biohadoop achieve good speedups. Those properties make Biohadoop a good match for the implementation and execution of parallelized bio-inspired optimization techniques on Hadoop.
% 
% 
% The evaluation results show that algorithms suited to parallelization can achieve good speedups when implemented with Biohadoop. Using Hadoop for cluster management and reliable application execution, 
% 
% 
% 
% In conclusion, the evaluation results show that Biohadoop is a useful framework for the implementation of bio-inspired optimization techniques. It provides a simple and powerful non-blocking API for distributed computations. It was demonstrated that algorithms suited to parallelization can achieve good speedups. This is accomplished by 
% 
%  and . Biohadoop relies on Hadoop for cluster management and reliable execution of parallelized algorithms
% Biohadoop uses Hadoop for reliable execution of parallelized algorithms on a cluster. It was demonstrated that algorithms suited to parallelization can achieve good speedups. This properties make Biohadoop a good match for the implementation and execution of parallelized bio-inspired optimization techniques on Hadoop.
% 
% 
% Given that an algorithm is well suited for parallelization, Biohadoop offers an easy way for its implementation. The evaluation demonstrated also that such parallelized algorithms are able to perform very well on Hadoop.
% 
% In conclusion, the evaluation results show that Biohadoop is a useful framework for the implementation of bio-inspired optimization techniques. The underlying Hadoop system provides with HDFS and YANR a solid basis for the execution. Biohadoop provides a simple but powerful non-blocking API for distributed computations
% 
% 
% 
% In conclusion, the evaluation results show that Biohadoop is a useful framework for the implementation of bio-inspired optimization techniques on Hadoop. It provides a simple but powerful non-blocking API for distributed computations that otherwise needs to be implemented manually. Hadoop as basis layer simplifies cluster usage and reliable application execution.