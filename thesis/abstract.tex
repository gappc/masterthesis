Problem optimization is a fundamental task we encounter everywhere, from everyday life to the most complex science areas. Finding the optimal solution often takes an unreasonable amount of time or computing resources, therefore approximation techniques are used to find near-optimal solutions. Bio-inspired algorithms provide such approximation techniques, they are based on existing solutions found in the nature. But even those techniques are sometimes to slow for extensive problems, so they need to be run in parallel.

In this master-thesis, implementations of some bio-inspired optimization techniques are provided, that can be run on an Apache Hadoop cluster, by using the capabilities of YARN. The runtimes of those algorithms are then compared to their sequential version. Finally, the implementations are made usable by Apache Oozie, which is a Hadoop workflow scheduler that uses XML for its workflow configuration. This way, those optimization techniques are made accessible to a broader range of users.