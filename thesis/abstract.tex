Problem optimization is a fundamental task encountered everywhere, from everydays life to the most complex science areas. Finding the optimal solution often takes an unreasonable amount of time or computing resources. Therefore, approximation techniques are used to find near-optimal solutions. Bio-inspired algorithms provide such approximation techniques, they mimic existing solutions found in the nature. But even those techniques are sometimes to slow for extensive problems, so they need to be run in parallel.

This master thesis presents a new framework, Biohadoop, to facilitate the implementation and execution of parallelized bio-inspired optimization techniques on Apache Hadoop. Its usefulness is demonstrated by the implementation and performance evaluation of two bio-inspired optimization algorithms. Finally, an extension to the workflow scheduler Apache Oozie is presented that simplifies the usage of Biohadoop as part of a larger workflow.

% In this master-thesis, implementations of some bio-inspired optimization techniques are provided that can be run on an Apache Hadoop cluster, by using the capabilities of YARN. The runtimes of those algorithms are then compared to their sequential version. Finally, the implementations are made usable by Apache Oozie, which is a Hadoop workflow scheduler that uses XML for its workflow configuration. This way, those optimization techniques are made accessible to a broader range of users.