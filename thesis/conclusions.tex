\chapter{Conclusions}
\label{chap:conclusions}
This thesis presented Biohadoop, a new framework to build bio-inspired optimization techniques executable on Apache Hadoop. Two different GAs were implemented on top of Biohadoop. Their performance was evaluated on a Hadoop cluster with 6 dual-core computers that were connected to a \unit[1]{Gb} Ethernet network.

The results show that algorithms implemented with Biohadoop have the potential to scale efficiently if the parallel part of the problem dominates the whole execution time. An example of such a problem is the tiled matrix multiplication (TMM). It provided speedups of up to 10 when compared to the execution with less resources (one worker). ZDT-3 as the other benchmark exposed a behavior that lead to poor speedups of about 2. The problem with this benchmark was that the parallel running parts were very short in terms of computational time. The sequential time and communication overhead dominated the whole runtime.

The execution time comparison with standalone, sequential implementations demonstrated further advantages of Biohadoop. TMM provided in this case speedups of about 8. In contrast, the performance of ZDT-3 was worse to the point that the parallel execution took longer than the sequential version.

In conclusion, Biohadoop demonstrated to be a useful framework for the implementation of bio-inspired optimization techniques on Hadoop. It provides a powerful non-blocking API that simplifies the design of distributed applications. Hadoop as solid and well tested basis provides necessary cluster management capabilities and reliable application execution. Further, the evaluation results show that algorithms suited to parallelization and implemented with Biohadoop achieve good speedups. Those properties make Biohadoop a good match for the implementation and execution of parallelized bio-inspired optimization techniques on Hadoop.