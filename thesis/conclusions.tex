\chapter{Conclusions}
\label{chap:conclusions}
This thesis presented Biohadoop, a new framework to build bio-inspired optimization techniques executable on Apache Hadoop. Two different GAs were implemented on top of Biohadoop. Their performance was evaluated on a Hadoop cluster with 6 dual-core computers that were connected to a \unit[1]{Gb} Ethernet network.

The results show that algorithms implemented with Biohadoop have the potential to scale efficiently if the parallel part of the problem dominates the whole execution time. An example of such a problem is the tiled matrix multiplication (TMM). It provided speedups of up to 10 when compared to the execution with less resources (one worker). ZDT-3 as the other benchmark exposed a behavior that lead to poor speedups of about 2. The problem with this benchmark was that the parallel running parts were very short in terms of computational time. The sequential time and communication overhead dominated the whole runtime.

The execution time comparison with standalone, sequential implementations demonstrated further advantages of Biohadoop. TMM provided in this case speedups of about 8. In contrast, the performance of ZDT-3 was worse to the point that the parallel execution took longer than the sequential version.

% Biohadoop demonstrated during the ZDT-3 benchmark that it is capable of saturating a \unit[1]{Gb} Ethernet network. The achieved data rates peaked at about \unit[900]{Mb/s} which is close to the theoretical limit of the underlying network. However, the achieved data rate depends on the size of exchanged message.

In conclusion, the evaluation results show that Biohadoop is a useful framework for the implementation of bio-inspired optimization techniques. Given that an algorithm is well suited for parallelization, Biohadoop offers an easy way for its implementation. The evaluation demonstrated also that such parallelized algorithms are able to perform very well on Hadoop.